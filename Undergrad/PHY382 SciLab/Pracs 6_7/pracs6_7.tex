\documentclass[11pt]{article}

\usepackage{geometry}
\geometry{margin=2cm}
\usepackage{graphicx}
\usepackage{hyperref}
\usepackage{amsfonts}
\usepackage{caption}
\usepackage{subcaption}

\usepackage{amsthm}
% for defining definitions
\newtheorem{defn}{Definition}


\hypersetup{colorlinks=true, linkcolor=blue, urlcolor=blue}
\urlstyle{same}
\begin{document}
	
	\author{Aayush Arya}
	\date{(Submitted: \today)}
	\title{}
	
	\maketitle
	
	\hrule
	\begin{center}
		PHY382 Lab Report\\
		Practical(s): 6 \& 7 \quad Registration No.: 11912610 \quad Section: G2903
	\end{center}
	\hrule
	
	\section*{Laplace Transforms}
	There are functions $f(t)$ for which a Fourier transform is not always possible, as the function might not have a finite norm (i.e. $\int_{-\infty}^\infty |f(t)| dt$ could be divergent). However, it's very much possible that for the same function, $\int f(t)e^{-st}dt$ could still be finite. 
	
	The Laplace transform, is an integral transform that exploits simply this. 

	
	\begin{defn}
		The {\it Laplace transform} of a function $f(t)$ over a transform variable $s \in \mathbb{C}$ is defined as \[ F(s) = \mathcal{L}(f; s) := \int_0^\infty f(t) e^{-st} dt\] 
	\end{defn}
	Note that this is a single-sided Laplace transform where the integral runs from $0$ to $\infty$. However, a two-sided Laplace transform can also be defined.
	
	Because $e^{-st}$ is a rapidly declining function of $t$, for a $f(t)$ that is a polynomial function in $t$, the product $f(t)e^{-st}$ will have a finite integral over $[0, \infty)$. In fact, even if $f$ is exponential type, that is $f(t) = e^{at}$, one can still find a Laplace tranform given that $Re(s) > a$.\\
	
	Therefore, it's possible to find the Laplace transform of a function in cases when a Fourier transform is not possible.
	
	\subsection*{Applications}
	It may be asked why these transforms are useful. The utility of these transforms is that $\mathcal{L}$ is a {\it linear} transformation and it's possible to find an inverse Laplace transform later on.
	
	A very interesting utility is in solving differential equations. The Laplace transform of a derivative of $f(t)$ turns out to become a polynomial in $s$.  
	
	\subsection*{LCR Circuits}
	The differential equation governing the current in an LCR circuit is
	
	\[L\frac{d^2q}{dt^2} + R\frac{dq}{dt} + \frac{q}{C}= V(t)\]
	
	For a moment let's write $\mathcal{L} (q;s) = Q(s)$. Using the fact that \[\mathcal{L}\left( \frac{d^nf}{dt^n} \right) = -s^nF(s) + \sum_{k=1}^n s^{n-k}f^{k-1 }(0)\]
	
	We can write \[\mathcal{L} \left( \frac{d^2 q}{dt^2}\right) = -s^2 Q(s) + sq(0) + \frac{dq}{dt}(0)	\]
	
	and \[ \mathcal{L} (\frac{dq}{dt}) = -sQ(s) +  q(0)\]
	Using these two, for the {\it free oscillation} case ($V=0$) the differential equation becomes
	
	\[ -s^2Q(s) + sq(0) + \frac{dq}{dt}(0) - sQ(s) + q(0) = 0\]
	
	rearranging, we have
	 \[ \frac{dq}{dt}(0) + (s + 1)q(0)  - s(s+1)Q(s) = 0\]
	 
	 After which we require initial conditions \textemdash namely the initial current (e.g. $I(0) = 0$) and charge. Say, for instance, if $I(0) = dq/dt(0) = 0$,
	 
	 \[Q(s) = \frac{s+1}{s(s+1)}q(0)\]
	 
	 And then we can take the inverse Laplace transform of $Q(s)$ to recover $q(t)$.
\end{document}