\documentclass[11pt, a4paper]{article}

\usepackage{amsmath}
\usepackage{amsfonts}
\usepackage{physics}
\usepackage{geometry}
\usepackage{graphicx}

\geometry{top=0cm, margin=1.5cm}
\begin{document}
	\author{Aayush Arya}
	\title{Determining the wavelength of H-$\alpha$ emission}
	\maketitle
	
	\hrule
	\begin{center}
		PHY259 Lab Report\\
		Registration Number: 11912610 \quad Section: G2903
	\end{center}
	\hrule
	
	\section*{Measurements \& Calculations}
	
	A mercury lamp was used as the light source. The number of grating lines $N$ per unit length is taken to be $600$ mm$^{-1}$. The true wavelength of H-$\alpha$ known to us is $656.281$ nm.
	
	
	The angular measurements were the following.
	
	\begin{table}[h]
		\centering
		\begin{tabular}{|c||c|c|c|c|c|}
			\hline
			Position & V1 & & V2 & & Difference \\
			\hline
			& Main Scale & Vernier Scale & Main Scale & Circular Scale & \\
			\hline
			Left & 336 & 18 & 156& 18 & 180 \\
			Right & 23 & 14 & 203 & 14 & 180 \\
			\hline
		\end{tabular}
		\caption{Position of H-$\alpha$ for the left and right first orders of diffraction. The main scale readings are in degrees, and each vernier scale division is equivalent to 2'.}
	\end{table}
	
	The difference between the left and the right is of $2\theta = 46^o52'$. This gives the angular position of H-$\alpha$ in the first order from the center to be $\theta = 23^o26'$.
	
	The grating equation is $$ d\sin \theta = m\lambda$$
	Since $d = 1/N$ where $N$ is line density, in SI units $N=6\times10^5$ per metres gives $d = 1.67 \times 10^{-6}$ For first order, $m=1$
	
	$$ (1.67 \times 10^{-6})\times \sin (23^o26')= \lambda$$
	$$ \implies \lambda = 6628.02 \text{\AA}$$
	\section*{Results}
	The measured wavelength of H-$\alpha$ is $6628.02$\AA, which is $45.21$\AA off the true value of $6562.81$\AA, giving a percentage deviation of $0.69\%$ 
	
\end{document}