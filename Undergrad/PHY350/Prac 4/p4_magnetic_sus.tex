\documentclass{article}

\usepackage{geometry}
\geometry{margin=2cm}
\usepackage{graphicx}
\usepackage{hyperref}
\usepackage{amsfonts}

\hypersetup{colorlinks=true, linkcolor=blue, urlcolor=blue, citecolor=blue}
\urlstyle{same}
\begin{document}
	
	\author{Aayush Arya}
	\date{(Submitted: \today)}
	\title{}
	
	\maketitle
	
	\hrule
	\begin{center}
		PHY350 Lab Report\\
		Practical: 4 \quad Registration No.: 11912610 \quad Section: G2903
	\end{center}
	\hrule
	
	\section*{Aim}
	To measure the magnetic susceptibility of a paramagnetic material using Quinke's method
	
	\section*{Methods}
	
	Density of the liquid $\rho = 1.443$ g $cm^{-3}$
	  
	  $$ \chi = \frac{2 \rho g h}{\mu_0 H^2} = \frac{2\mu_0 \rho g h}{B^2}$$
	  
	Value for $H$ with varying $I$ was noted. Further, the height level of the liquid in the capillary tube was noted.
	
	\section*{Results}
	
	\begin{table}[h]
		\centering
		\begin{tabular}{|c|c|c|c|c|}
			\hline
			I (A)  & B & MSR   & VSR & Total (cm)  \\
			\hline
			0.5 & 0.113 & 12.85 & 2.0 & 12.8520 \\
			1.5 & 0.340 & 12.85 & 2.4 & 12.8524 \\
			2.5 & 0.567 & 12.85 & 3.2 & 12.8532 \\
			3.5 & 0.793 & 12.85 & 4.2 & 12.8542\\
			\hline
		\end{tabular}
	\end{table}

	A problem in calculating `h' was that the zero-point the height measurement is not known. By manual extrapolating from the given data points, it was inferred that the zero point should be very close to 12.8520 cm.\\
	
	Calculating gave an average value of $\chi = 1.26\times 10^{-6}$.
	
\end{document}